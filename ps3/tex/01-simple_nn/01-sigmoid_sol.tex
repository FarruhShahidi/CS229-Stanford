\begin{answer}
 Recall $\sigma$ denotes the sigmoid function and for computations $\sigma' = \sigma (1 - \sigma)$ is handy.
 Write $h_i = \sigma(\omega _{1,i}^{[1]}x_1 + \omega _{2,i}^{[1]}x_2 + b_j^{[1]}) (1)$ for $j = 1,2 ,3$ and $o = \sigma(\omega _{1}^{[2]}h_1 + \omega _{2}^{[2]}h_2 + b^{[2]})$
The graduate descent update for $\omega _{1,2}^{[1]}$ is
$$\omega _{1,2}^{[1]} = \omega _{1,2}^{[1]} - \alpha \frac{\partial l}{\partial \omega _{1,2}^{[1]}}.$$
Now we  will evaluate the partial derivate using the chain rule and by noting that in order to evaluate the desired partial
derivative we only need to look at the function $h_2.$

$$\frac{\partial l}{\partial \omega _{1,2}^{[1]}}= \frac 2m\sum_{i}(o^{(i)} - y^{(i)})\frac{\partial o^{(i)}}{\partial \omega _{1,2}^{[1]}}$$
$$ = \frac 2m\sum_{i}(o^{(i)} - y^{(i)}) o^{(i)}(1 - o^{(i)})\omega_2^{[2]}h_2(1 - h_2)x^{(i)}.$$ 

Although, the final outcome is expressed in terms of the function $h_2$ one can easily replace it with using (1) and putting the superscript $(i).$      
 \end{answer}

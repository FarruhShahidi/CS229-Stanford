\clearpage
\item \subquestionpoints{7} For this part of the problem only, you may
  assume $n$ (the dimension of $x$) is 1, so that $\Sigma = [\sigma^2]$ is
  just a real number, and likewise the determinant of $\Sigma$ is given by
  $|\Sigma| = \sigma^2$.  Given the dataset, we claim that the maximum
  likelihood estimates of the parameters are given by
  \begin{eqnarray*}
    \phi &=& \frac{1}{m} \sum_{i=1}^m 1\{y^{(i)} = 1\} \\
\mu_{0} &=& \frac{\sum_{i=1}^m 1\{y^{(i)} = {0}\} x^{(i)}}{\sum_{i=1}^m
1\{y^{(i)} = {0}\}} \\
\mu_1 &=& \frac{\sum_{i=1}^m 1\{y^{(i)} = 1\} x^{(i)}}{\sum_{i=1}^m 1\{y^{(i)}
= 1\}} \\
\Sigma &=& \frac{1}{m} \sum_{i=1}^m (x^{(i)} - \mu_{y^{(i)}}) (x^{(i)} -
\mu_{y^{(i)}})^T
  \end{eqnarray*}
  The log-likelihood of the data is
  \begin{eqnarray*}
\ell(\phi, \mu_{0}, \mu_1, \Sigma) &=& \log \prod_{i=1}^m p(x^{(i)} , y^{(i)};
\phi, \mu_{0}, \mu_1, \Sigma) \\
&=& \log \prod_{i=1}^m p(x^{(i)} | y^{(i)}; \mu_{0}, \mu_1, \Sigma) p(y^{(i)};
\phi).
  \end{eqnarray*}
By maximizing $\ell$ with respect to the four parameters,
prove that the maximum likelihood estimates of $\phi$, $\mu_{0}, \mu_1$, and
$\Sigma$ are indeed as given in the formulas above.  (You may assume that there
is at least one positive and one negative example, so that the denominators in
the definitions of $\mu_{0}$ and $\mu_1$ above are non-zero.)

\ifnum\solutions=1 {
  \begin{answer}
The product and the sums run from  1 to m, so for simplicity I drop the limits. Note that
$$l = \log\prod p(y^{(i)}| x^{(i)})p(y{(i)}) = \sum\log p(y^{(i)}| x^{(i)})p(y{(i)}) = $$$$
 -m\log(\sigma) + c_0 - \sum\big( y^{(i)}\frac{(x^{(i)} - \mu_0)^2}{2\sigma^2} + (1- y^{(i)})\frac{(x^{(i)} - \mu_1)^2}{2\sigma^2}\big) + \sum(y^{(i)}\phi + (1-y^{(i)})(1-\phi)).$$
 
 Now we calculate the partial derivatives.
 
 $$\frac{\partial}{\partial\phi} = \sum(\frac{y^{(i)}}{\phi} + \frac{1- y^{(i)}}{1-\phi}) = 0.$$
 The numerator after the sum is then $\sum y^{(i)} - m\phi$ and so
 $$\phi= \frac{\sum y^{(i)}}{m}.$$
 Not that the formula is exactly as the formula given in the problem because if $y^{(i)} = 0$ the $i-$th term of the sum is $0.$ Hence this formula can be represented with the characteristic function. I will not be re-writing this as well as the problems below as I do not know how useful it may be in the coding part.
 Further derivatives (enough to show $\mu_0$):
 $$\frac{\partial l}{\partial\mu_0} =\sum\frac{y^{(i)}(x^{(i)} - \mu_0 )}{\sigma^2} = \frac{\sum x^{(i)}y^{(i)} - \mu_0\sum y^{(i)}}{\sigma^2}.$$
 Hence we obtain
 $$\mu_0 = \frac{\sum x^{(i)}y^{(i)}}{\sum y^{(i)}}.$$
 Finally, 
 $$\frac{\partial l}{\partial \sigma} = \frac{m}{\sigma} - \frac{1}{\sigma^3}(\sum\big( y^{(i)}\frac{(x^{(i)} - \mu_0)^2}{2\sigma^2} + (1- y^{(i)})\frac{(x^{(i)} - \mu_1)^2}{2\sigma^2}\big)).$$
 By solving this equation for $\sigma$ we obtain the desired result.
 
 
 
\end{answer}

} \fi

\clearpage
\item \subquestionpoints{5}
Recall that in GDA we model the joint distribution of $(x, y)$ by the following
equations:
%
\begin{eqnarray*}
	p(y) &=& \begin{cases}
	\phi & \mbox{if~} y = 1 \\
	1 - \phi & \mbox{if~} y = 0 \end{cases} \\
	p(x | y=0) &=& \frac{1}{(2\pi)^{n/2} |\Sigma|^{1/2}}
		\exp\left(-\frac{1}{2}(x-\mu_{0})^T \Sigma^{-1} (x-\mu_{0})\right) \\
	p(x | y=1) &=& \frac{1}{(2\pi)^{n/2} |\Sigma|^{1/2}}
		\exp\left(-\frac{1}{2}(x-\mu_1)^T \Sigma^{-1} (x-\mu_1) \right),
\end{eqnarray*}
%
where $\phi$, $\mu_0$, $\mu_1$, and $\Sigma$ are the parameters of our model.

Suppose we have already fit $\phi$, $\mu_0$, $\mu_1$, and $\Sigma$, and now
want to predict $y$ given a new point $x$. To show that GDA results in a
classifier that has a linear decision boundary, show the posterior distribution
can be written as
%
\begin{equation*}
	p(y = 1\mid x; \phi, \mu_0, \mu_1, \Sigma)
	= \frac{1}{1 + \exp(-(\theta^T x + \theta_0))},
\end{equation*}
%
where $\theta\in\Re^n$ and $\theta_{0}\in\Re$ are appropriate functions of
$\phi$, $\Sigma$, $\mu_0$, and $\mu_1$.

\ifnum\solutions=1{
  \begin{answer}
Apply Bayes
$$p(y = 1| x) = \frac{p(x| y=1)p(y = 1)}{p(x| y=1)p(y = 1) + p(x| y=0)p(y = 0)} = \frac{1}{1 + \frac{p(x| y=0)p(y = 0)}{p(x| y=1)p(y = 1)}}.$$
Note that $\frac{p(y = 0)}{p(y = 1)} = exp(ln(\frac{1-\phi}{\phi})$
For simplicity  denote A = $\Sigma^{-1}.$
Note that
$$\frac{p(x| y=0)}{p(x| y=1)} = exp(-\frac{1}{2}(A(x- \mu_0), x - \mu_0) - (A(x- \mu_1), x - \mu_1))) $$$$= exp(-\frac 12((Ax,x) - 2(Ax, \mu_0) + (\mu_0, \mu_0) - (Ax,x) + 2(Ax, \mu_1) + (\mu_1, \mu_1)))$$
$$= exp(-( -(Ax, \mu_0) + \frac 12(A\mu_0, \mu_0) + (Ax, \mu_1) - \frac12(A\mu_1, \mu_1))) = exp(-(\mu_1 - \mu_0)^TAx + \frac{1}{2}((A\mu_0,\mu_0) - (A\mu_1,\mu_1)).$$
From this we find $\theta^T = (\mu_1 - \mu_0)^TA$ hence 
$$\theta = (\Sigma^{-1})^T(\mu_1 - \mu_0)$$
and
$$\theta_0 = \frac{1}{2}((\Sigma^{-1}\mu_0,\mu_0) - (\Sigma^{-1}\mu_1,\mu_1)) - ln(\frac{1-\phi}{\phi}).$$

\end{answer}

}\fi
